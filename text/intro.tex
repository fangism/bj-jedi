% "intro.tex"

% \nextpageodd
\chapter{Introduction}
\label{sec:intro}

Blackjack (or 21) is a game played by one or more individuals against
a dealer, an opponent whose actions are predetermined.
No player's decisions will influence the dealer's decision.  
The winner of each hand is whoever (player or dealer)
has a card total closest to, but not exceeding, 21.  
This seemingly simple game has captivated millions of gamblers
and card game enthusiasts for centuries, 
and continues to be popular today.

This text details a raw, mathematical analysis of the game of Blackjack.
The first sections derived many well-known results of Blackjack, 
including optimal basic strategy and house edge.

% new insights?

% what is NOT in this book?

\section{History}
\label{sec:intro:history}

\section{Rules and Gameplay}
\label{sec:intro:rules}

Blackjack can be played with one or more standard decks of cards.
Cards numbered 2 thorugh 10 are worth their face value.  
Face cards (Jack, Queen, King) are worth 10.
Aces are worth 1 or 11, which is up to the player.
Aces are worth 11 when the total does not exceed 21.
For the remainder of this text, `10' shall refer to any 10-valued card.
An Ace and a 10 form the highest-valued hand, a blackjack, 
also known as a natural\index{natural}.
Thus in standard decks of cards, there are 4 times as many 10-valued 
cards as there are any of the other cards.

Blackjack has many simple variations around the world.
We begin with an outline one of the most common variations of the game.
A typical hand of Blackjack is played as follows:

\begin{enumerate}
\item The player places a bet on the table for the next hand, 
before any cards are dealt.
\item Cards are dealt: 2 to the player (face-up), 
2 to the dealer (one revealed, one hidden, called the hole card)
\item Checking for ``\emph{natural}'' blackjack by player or dealer\index{natural}:
\begin{enumerate}
\item If the dealer reveals an Ace, the player is offered insurance\index{insurance},
a side-bet of whether or not the hole card is a 10 (giving the dealer a natural).
The typical side-bet amount is half of the original bet.
Regardless of the insurance outcome, the player loses her original bet
if the dealer has blackjack (and the player does not).
If the dealer does not have blackjack, play resumes.  
\item If the player has a natural (first two cards total 21), 
and the dealer does not, then the player automatically wins
(typically 1.5 times her original bet).
\item If both the dealer and player have natural blackjack, it is considered
a push\index{push} (tie), and the bet is neither won nor lost.  
\end{enumerate}
\item Only with the first 2 cards, the player is given the option to:
\begin{enumerate}
\item \textbf{Double-down}:\index{double-down} player takes exactly 1 more card 
and doubles her bet
\item \textbf{Split}:\index{split} turn paired hand into two hands, 
each of which is dealt an additional up-card.  
Each new hand is then played normally.
A total of 21 on any of the newly split hands is treated as 21, 
not a natural blackjack.
Sometimes the player is permitted to double-down on the newly split hands.
Sometimes the player is permitted to re-split paired hands further.
Paired Aces often have restrictions.  
\item \textbf{Surrender}:\index{surrender} give up the hand by losing half of 
the original bet (also known as late surrender).
\item If none of these options are exercised, the hand resumes normally.
\end{enumerate}
\item The player elects to hit (take additional cards) until either
she busts (total exceeds 21) or stands, taking no more cards.
In the case of a player bust, the bet is lost immediately.
\item Once the player stands, the dealer hits until his hand either busts
or his cards' total is 17 or higher.  
If the dealer busts, then the player wins the bet immediately.  
\item Whoever has the total closest to 21 wins the hand.
In the event of a push (tie), the bet is null.
\end{enumerate}

% hand signals when playing at a live table

In the presence of multiple players playing against the dealer, 
all of the players act first before the dealer reveals the hole card
and plays.  
Players win or lose only against the dealer, not against each other.
The presence of multiple players does not impact the overall odds of the game.

Table~\ref{tab:win-lose} summarizes the outcome of a player's hand total
against the dealer's hand.
Since the dealer never stands below 17, all player hands $<=$ 16
are considered equal; they can only win when the dealer busts.
`BJ' is the natural or blackjack state, which is determined 
before any player actions are allowed.
`Push' is a special dealer state for a variant called Blackjack Switch, 
discussed later.

\begin{table}[ht!]
\caption{Player vs. dealer final state showdown}
\begin{center}
\input tables/win-lose-push
\end{center}
\label{tab:win-lose}
\end{table}

