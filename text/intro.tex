% "intro.tex"

% \nextpageodd
\chapter{Introduction}
\label{sec:intro}

Blackjack (or 21) is a game played by one or more individuals against
a dealer, an opponent whose actions are predetermined.
No player's decisions will influence the dealer's decision.  
The winner of each hand is whoever (player or dealer)
has a card total closest to, but not exceeding, 21.  
This seemingly simple game has captivated millions of gamblers
and card game enthusiasts for centuries, 
and continues to be popular today.

This text details a raw, mathematical analysis of the game of Blackjack.
The first sections derived many well-known results of Blackjack, 
including optimal basic strategy and house edge.

% new insights?

% what is NOT in this book?

%%%%%%%%%%%%%%%%%%%%%%%%%%%%%%%%%%%%%%%%%%%%%%%%%%%%%%%%%%%%%%%%%%%%%%%%%%%%%%%
\section{History}
\label{sec:intro:history}

%%%%%%%%%%%%%%%%%%%%%%%%%%%%%%%%%%%%%%%%%%%%%%%%%%%%%%%%%%%%%%%%%%%%%%%%%%%%%%%
\section{Rules and Gameplay}
\label{sec:intro:rules}

Blackjack can be played with one or more standard decks of cards.
Cards numbered 2 thorugh 10 are worth their face value.  
Face cards (Jack, Queen, King) are worth 10.
Aces are worth 1 or 11, which is up to the player.
Aces are worth 11 when the total does not exceed 21.
For the remainder of this text, `10' shall refer to any 10-valued card.
An Ace and a 10 form the highest-valued hand, a blackjack, 
also known as a natural\index{natural}.
Thus in standard decks of cards, there are 4 times as many 10-valued 
cards as there are any of the other cards.

Blackjack has many simple variations around the world.
We begin with an outline one of the most common variations of the game.
A typical hand of Blackjack is played as follows:

\begin{enumerate}
\item The player places a bet on the table for the next hand, 
before any cards are dealt.
\item Cards are dealt: 2 to the player (face-up), 
2 to the dealer (one revealed, one hidden, called the hole card)
\item Checking for ``\emph{natural}'' blackjack by player or 
dealer\index{natural}:
\begin{enumerate}
\item If the dealer reveals an Ace, the player is offered 
insurance\index{insurance},
a side-bet of whether or not the hole card is a 10 
(giving the dealer a natural).
The typical side-bet amount is half of the original bet.
Regardless of the insurance outcome, 
if the dealer has blackjack (and the player does not),
the player loses her original bet and the hand is over.
If the dealer does not have blackjack, play resumes.  
\item If the player has a natural (first two cards total 21), 
and the dealer does not, then the player automatically wins
(typically 1.5 times her original bet), and the hand is over.
\item If both the dealer and player have natural blackjack, 
it is considered a push\index{push} (tie), 
and the hand is over with the bet cancelled.  
\end{enumerate}
\item If the hand is not yet over, 
the player is assured that the dealer does not have blackjack, 
because he has already `peeked' for blackjack\index{peek}.
\item Only with the first 2 cards, the player is given the option to:
\begin{enumerate}
\item \textbf{Double-down}:\index{double-down} player takes exactly 1 more card 
and doubles her bet
\item \textbf{Split}:\index{split} turn paired hand into two hands, 
each of which is dealt an additional up-card.  
Each new hand is then played normally.
A total of 21 on any of the newly split hands is treated as 21, 
not a natural blackjack.
Sometimes the player is permitted to double-down on the newly 
split hands (double-after-split\index{double-after-split}).
Sometimes the player is permitted to re-split paired hands further.
Paired Aces often have restrictions.  
\item \textbf{Surrender}:\index{surrender} give up the hand by losing half of 
the original bet (also known as late surrender).
\item If none of these options are exercised, the hand resumes normally.
\end{enumerate}
\item The player elects to hit (take additional cards) until either
she busts (total exceeds 21) or stands, taking no more cards.
In the case of a player bust, the bet is lost immediately.
\item Once the player stands, the dealer hits until his hand either busts
or his cards' total is 17 or higher.  
If the dealer busts, then the player wins the bet immediately.  
\item Whoever has the total closest to 21 wins the hand.
In the event of a push (tie), the bet is null.
\end{enumerate}

% hand signals when playing at a live table

In the presence of multiple players playing against the dealer, 
all of the players act first before the dealer reveals the hole card
and plays.  
Players win or lose only against the dealer, not against each other.
The presence of multiple players does not impact the overall 
odds of the game; nor do their decisions impact your odds.

\begin{table}[ht!]
\caption{Player vs. dealer final state showdown}
\begin{center}
\input tables/win-lose-push
\end{center}
\label{tab:win-lose}
\end{table}

Table~\ref{tab:win-lose} summarizes the outcome of a player's hand total
against the dealer's hand.
Since the dealer never stands below 17, all player hands $<=$ 16
are considered equal; they can only win when the dealer busts.
`BJ' is the natural or blackjack state, which is determined 
before any player actions are allowed.
`Push' is a special dealer state for a variant called Blackjack Switch, 
discussed later.

When either the player or dealer has a natural
in the first two cards, the hand ends immediately
and there are no further decisions to be made.
The interesting part of the game is the rest of the time when
there are actions to be decided.
The player's decision to hit, double-down, and split
result in a probabilistic change of state, for better or worse.
By computing expectations\index{expectations} in every state, 
we can compute the optimal decision in every situation.  
The analysis of the expectations begins with
the dealer, who must act in accordance to his own hand only.

%%%%%%%%%%%%%%%%%%%%%%%%%%%%%%%%%%%%%%%%%%%%%%%%%%%%%%%%%%%%%%%%%%%%%%%%%%%%%%%
\section{Dealer's Play}
\label{sec:rules:dealer-play}

\begin{table}[ht!]
\caption{Dealer's state transition table (hits on soft-17)}
\begin{center}
\input tables/dealer-hit-H17
\end{center}
\label{tab:dealer-hit-H17}
\end{table}

\begin{table}[ht!]
\caption{Dealer's state transition table (stands on soft-17)}
\begin{center}
\input tables/dealer-hit-S17
\end{center}
\label{tab:dealer-hit-S17}
\end{table}

The blank rows represent the dealer's terminal states.
Since the dealer's actions are entirely independent of 
the players' actions, we can define the state-machine
for the dealer's cards, following the rule that the dealer
must hit until his total exceeds 16.
Table~\ref{tab:dealer-hit-H17} summarizes the various
value transitions that the dealer's hand may take
before reaching a terminal state.
The row headers contain the dealer's current hand value, 
and the column headers are the next card (if applicable).
The table entries point to the next state.

States labeled `A,$x$' represent soft-totals, where the Ace (A)
can be evaluated as 1 or 11.  
The terminal `A,$x$' states are treated as the higher of the 
two possible values, ranging from 17 to 21.  
In the first variation, Table~\ref{tab:dealer-hit-H17}, 
the dealer must hit on a soft-17 (A,6)\index{hit on soft-17}.
Later, we will compare the edge impact of this variation
vs. the dealer standing on soft-17\index{stand on soft-17}.
The corresponding state table for the stand-on-soft-17 variation
is shown in Table~\ref{tab:dealer-hit-S17}, 
the only difference being that the dealer stands on (A,6) 
as a terminal state.
As a shorthand, we refer to the hit-on-soft-17 variation as H17, 
and stand-on-soft-17 variation as S17.

In these two tables, the transition edges for A$\rightarrow$10
and 10$\rightarrow$A result in 21, and not a natural blackjack
because the dealer has already peeked\index{peek} for blackjack
with his first two cards (if an Ace or 10 was revealed).

By the state-transition process defined in these tables,
all non-terminal states will eventually reach
one of the terminal states (a.k.a. absorbing state), 
except for dealer blackjack (denoted BJ), 
which is be evaluated separately.
This state-transition process is also known as an
\emph{absorbing Markov chain}\index{Markov chain!absorbing},
since every state can reach an absorbing state.


\subsection{Probability of Dealer's Terminal States}
\label{sec:rules:dealer-play:final-pdf}

Given the dealer's revealed card, 
one can calculate the probability of the dealer ending up in each state.
At any given state, the probability distribution of the
next state is simply based on the card probabilities.  
The dealer's hit transition table is acyclic, i.e. 
no state can ever reach itself again.  
Thus, there exists a topological sort\index{topological sort}
ordering of states that can be used to efficiently and iteratively 
compute the final state probabilities.
The height of the topologically sorted graph is the maximum 
number of iterations needed to converge to terminal states.
For each initial card, the initial probability vector
is a unit vector with all entries set to 0.0, except for the state
that the initial card represents, which is set to 1.0.
At each iteration, the probabilities of each state are 
``redistributed''\index{redistribution} to the next states using the 
card probabilities as edge weights, whose sum is 1.0.  
At any given iteration, the sum of probabilities is always 1.0.
Probabilities of the terminal states never get redistributed, 
they just accumulate.  
The system converges once all non-terminal state probabilities 
in the probability vector are 0.0.

The process described above assumes an invariant
card distribution, when in fact, as each card is revealed and removed, 
the probability distribution of remaining cards changes.
Assuming a constant distribution is akin to modeling
the deck as infinite.
For now we proceed using the infinite deck approximation.  
(Perhaps in a later section, we compute the inaccuracy of
assuming an infinite deck vs. a finite deck, where one 
draws cards without replacement.)

% Pre-peek vs. Post-peek.

\begin{table}[ht!]
\caption{Dealer's final state distribution, H17, pre-peek}
\begin{center}
\input tables/dealer-final-H17-pre-peek
\end{center}
\label{tab:dealer-final-H17-pre-peek}
\end{table}

\begin{table}[ht!]
\caption{Dealer's final state distribution, H17, post-peek}
\begin{center}
\input tables/dealer-final-H17-post-peek
\end{center}
\label{tab:dealer-final-H17-post-peek}
\end{table}

Suppose that the dealer was given just one up-card with no hole card, 
and the dealer hit until reaching a terminal state.
Suppose further that if the first two cards are Ace and 10 is
treated as a natural blackjack (BJ).
Then the spread of final states would compute as 
Table~\ref{tab:dealer-final-H17-pre-peek}.
The entries are reported in percentages, so the columns add up to 100\%.

However, in most blackjack variations, if the revealed card is an Ace or 10, 
the dealer will check for blackjack by peeking\index{peek} at the hole card.
In the event of a dealer natural (or player natural), 
the dealer reveals the hole card (usually to the player's disappointment)
and the hand ends before the player is given the opportunity to act.
As soon as the player is prompted for action, she knows that the 
dealer cannot have a natural.
This means for a revealed Ace, the hole card cannot be 10, 
and for a revealed 10, the hole card cannot be Ace.
Since the first iteration of the redistribution\index{redistribution} 
calculation represents the hole card, the first iteration
should be computed by zero-ing out the probability of the
card that is eliminated from the possibilities, and re-normalizing
the probability of the other cards (to keep a sum of 1.0).  
The result is a ``post-peek'' spread of final states, computed as 
Table~\ref{tab:dealer-final-H17-post-peek}.
Note that only the Ace and 10 columns are differ from 
Table~\ref{tab:dealer-final-H17-pre-peek}, 
because the dealer need not peek for blackjack on all other up-cards.

\begin{table}[ht!]
\caption{Dealer's final state distribution, S17, pre-peek}
\begin{center}
\input tables/dealer-final-S17-pre-peek
\end{center}
\label{tab:dealer-final-S17-pre-peek}
\end{table}

\begin{table}[ht!]
\caption{Dealer's final state distribution, S17, post-peek}
\begin{center}
\input tables/dealer-final-S17-post-peek
\end{center}
\label{tab:dealer-final-S17-post-peek}
\end{table}

For the S17 variation, the corresponding tables are similarly computed.
The pre-peek distribution is computed as 
Table~\ref{tab:dealer-final-S17-pre-peek}, 
and the post-peek distribution is computed as 
Table~\ref{tab:dealer-final-S17-post-peek}.

% observations

When we compute the optimal basic strategy in Chapter~\ref{sec:basic}, 
we will use the post-peek tables, since only they are relevant
when the player has choice in action.  
Pre-peek tables are still useful for evaluating overall player or house
edges \emph{before} each hand is dealt.

%%%%%%%%%%%%%%%%%%%%%%%%%%%%%%%%%%%%%%%%%%%%%%%%%%%%%%%%%%%%%%%%%%%%%%%%%%%%%%%
\section{Player's State Transitions}
\label{sec:rules:player-hit}

\begin{table}[ht!]
\caption{Player's state transition table}
\begin{center}
\input tables/player-hit
\end{center}
\label{tab:player-hit}
\end{table}

Every `hit' by the player is also represented as a state transition.
Unlike the dealer, however, the player always has the option
to stand in any state.
Table~\ref{tab:player-hit} shows all of the valid changes of state
when taking another card, and is similar to the dealer's table.
This table just shows possible transitions, 
not all of them are necessarily favorable.

\begin{table}[ht!]
\caption{Player's final split transition table}
\begin{center}
\input tables/player-final-split
\end{center}
\label{tab:player-final-split}
\end{table}

\begin{table}[ht!]
\caption{Player's resplit transition table}
\begin{center}
\input tables/player-resplit
\end{center}
\label{tab:player-resplit}
\end{table}

For paired hands, the player is given the option to split the
hand into two new hands, with each receiving a new up-card on top
of the original split card.
If the player is only allowed to split to two hands
(this is her last exercisable split), 
then the transition table is Table~\ref{tab:player-final-split}.
If further splitting is permitted, then the state transition table
is Table~\ref{tab:player-resplit}.

