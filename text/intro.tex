% "intro.tex"

% \nextpageodd
\chapter{Introduction}

Blackjack (or 21) is a game played by one or more individuals against
a dealer, an opponent whose actions are predetermined.
No player's decisions will influence the dealer's decision.  
The winner of each hand is whoever (player or dealer)
has a card total closest to, but not exceeding, 21.  
This seemingly simple game has captivated millions of gamblers
and card game enthusiasts for centuries, 
and continues to be popular today.

This text details a raw, mathematical analysis of the game of Blackjack.
The first sections derived many already-known results of Blackjack, 
including optimal basic strategy and house edge.

\section{History}

\section{Rules}

Blackjack has many simple variations around the world.
First, we outline one of the most common variations of the game.

A typical hand of Blackjack is played as follows:

\begin{enumerate}
\item The player places a bet on the table for the next hand, 
before any cards are dealt.
\item Cards are dealt: 2 to the player (face-up), 
2 to the dealer (one revealed, one hidden, called the hole card)
\item Checking for ``natural'' blackjack by player or dealer:
\begin{enumerate}
\item If the dealer reveals an Ace, the player is offered insurance,
a side-bet of whether or not the hole card is a 10 (giving the dealer a natural).
The typical side-bet amount is half of the original bet.
Regardless of the insurance outcome, the player loses her original bet
if the dealer has blackjack (and the player does not).
If the dealer does not have blackjack, play resumes.  
\item If the player has a natural (first two cards total 21), 
and the dealer does not, then the player automatically wins
(typically 1.5 times her original bet).
\item If both the dealer and player have blackjack, it is considered
a push (tie), and the bet is neither won nor lost.  
\end{enumerate}
\item TODO: double-down, split, surrender
\item The player elects to hit (take additional cards) until either
she busts (total exceeds 21) or stands, taking no more cards.
In the case of a player bust, the bet is lost immediately.
\item Once the player stands, the dealer hits until his hand either busts
or his cards' total is 17 or higher.  
If the dealer busts, then the player wins the bet immediately.  
\item Whoever has the total closest to 21 wins the hand.
In the event of a push (tie), the bet is null.
\end{enumerate}

Multiple players.

