% "intro.tex"

% \nextpageodd
\chapter{Introduction}
\label{sec:intro}

Blackjack (or 21) is a game played by one or more individuals against
a dealer, an opponent whose actions are predetermined.
No player's decisions will influence the dealer's decision.  
The winner of each hand is whoever (player or dealer)
has a card total closest to, but not exceeding, 21.  
A hand that exceeds 21 is called a
\emph{bust}\index{bust} and automatically loses.
This seemingly simple game has captivated millions of gamblers
and card game enthusiasts for centuries, 
and continues to be popular today.

This text details a raw, mathematical analysis of the game of Blackjack.
The first sections derived many well-known results of Blackjack, 
including optimal basic strategy and house edge.
To get the most out of this text, the reader should have a 
basic understanding of probability and matrix and vector arithmetic.

% new insights?

% what is NOT in this book?

%%%%%%%%%%%%%%%%%%%%%%%%%%%%%%%%%%%%%%%%%%%%%%%%%%%%%%%%%%%%%%%%%%%%%%%%%%%%%%%
\section{History}
\label{sec:intro:history}

% brief history

%%%%%%%%%%%%%%%%%%%%%%%%%%%%%%%%%%%%%%%%%%%%%%%%%%%%%%%%%%%%%%%%%%%%%%%%%%%%%%%
\section{Rules and Gameplay}
\label{sec:intro:rules}

Blackjack can be played with one or more decks of cards.
Cards numbered 2 thorugh 10 are worth their face value.  
Face cards (Jack, Queen, King) are worth 10.
Thus, in standard decks of cards, there are 4 times as many 10-valued 
cards as there are any of the other cards.
Aces are worth 11 when the total does not bust,
otherwise they are worth 1.
A \emph{hard} hand\index{hard total} is either a hand with no Aces,
or a hand with Aces that must be counted as 1.
A \emph{soft} hand\index{soft total} contains one Ace that can be 
counted as 11 without busting, for example, $A,5$ can count as 6 or 16.
Soft totals will always use the notation $A,x$.
For the remainder of this text, `10' shall refer to any 10-valued card.
An Ace and a 10 on the first two cards form the 
highest-ranking hand, a blackjack, 
also known as a natural\index{natural}.

Blackjack has many simple variations around the world.
We begin with an outline one of the most common variations of the game.
A typical hand of Blackjack is played as follows:

\begin{enumerate}
\item The player places a bet on the table for the next hand, 
before any cards are dealt.
\item Cards are dealt: 2 to the player (face-up), 
2 to the dealer (one revealed \emph{up-card}, one hidden \emph{hole card})
\item Checking for ``\emph{natural}'' blackjack by player or 
dealer\index{natural}:
\begin{enumerate}
\item If the dealer reveals an Ace, the player is offered 
insurance\index{insurance},
a side-bet of whether or not the hole card is a 10 
(giving the dealer a natural).
The typical side-bet amount is half of the original bet.
Regardless of the insurance outcome, 
if the dealer has blackjack (and the player does not),
the player loses her original bet and the hand is over.
If the dealer does not have blackjack, play resumes.  
\item If the player has a natural (first two cards total 21), 
and the dealer does not, then the player automatically wins
(typically 1.5 times her original bet), and the hand is over.
\item If both the dealer and player have natural blackjack, 
it is considered a push\index{push} (tie), 
and the hand is over with the bet cancelled.  
\end{enumerate}
\item If the hand is not yet over, 
the player is assured that the dealer does not have blackjack, 
because he has already `peeked' for blackjack\index{peek}.
\item Only with the first 2 cards, the player is given the option to:
\begin{enumerate}
\item \textbf{Double-down}:\index{double-down} player takes exactly 1 more card 
and doubles her bet, ending all of her actions.
\item \textbf{Split}:\index{split} divides a paired hand into two hands, 
each of which is dealt an additional up-card.  
Each new hand is then played normally.
A total of 21 on any of the newly split hands is treated as 21, 
not a natural blackjack.
Sometimes the player is permitted to double-down on the newly 
split hands (double-after-split\index{double-after-split}).
Sometimes the player is permitted to re-split paired hands further.
Paired Aces often have restrictions.  
\item \textbf{Surrender}:\index{surrender} give up the hand by losing half of 
the original bet (also known as late surrender), and keeping the other half.
\item If none of these options are exercised, the hand resumes normally.
\end{enumerate}
\item The player elects to hit (take additional cards) until either
she busts\index{bust} (total exceeds 21) or stands, taking no more cards.
In the case of a player bust, the bet is lost immediately.
\item Once the player stands, the dealer hits until his hand either busts
or his cards' total is 17 or higher.  
If the dealer busts, then the player wins the bet immediately.  
\item Whoever has the total closest to 21 wins the hand.
In the event of a push\index{push} (tie), no money is exchanged.
\end{enumerate}

% hand signals when playing at a live table

The presence of multiple players playing against the dealer does not
change the flow of the game.
Players win or lose only against the dealer, not against each other.
A dealer blackjack ends the hands for all players,
and players who do not also have blackjack lose their bet.
When the dealer does not have blackjack,
each player that has blackjack is paid off immediately, ending her hand, 
while the dealer continues to play against other live players.  
Every player act on her own hand before the dealer reveals the 
hole card and plays.  
The presence of multiple players does not impact the overall 
odds of the game; nor do their decisions impact your odds.

\begin{table}[ht!]
\caption{Player vs. dealer final state showdown}
\begin{center}
\input tables/win-lose-push
\end{center}
\label{tab:win-lose}
\end{table}

Table~\ref{tab:win-lose} summarizes the outcome of a player's hand total
against the dealer's hand.
Since the dealer never stands below 17, all player hands $<=$ 16
are considered equal; they can only win when the dealer busts.
`BJ' is the natural or blackjack state, which is determined 
before any player actions are allowed.
When the player has a natural blackjack against any non-blackjack 
dealer hand, she wins by the blackjack payoff amount, marked as `win*'
in Table~\ref{tab:win-lose}.  
`Push' is a special dealer state for a variant called Blackjack Switch, 
discussed later.
Note that player-BJ versus dealer-bust is somewhat irrelevant because
the dealer doesn't even get a chance to take more cards, 
except when playing against multiple players, 
in which case the outcome is the same.  

When either the player or dealer has a natural
in the first two cards, the hand ends immediately
and there are no further decisions to be made.
The interesting part of the game is the rest of the time when
there are actions to be decided.
The player's decision to hit, double-down, and split
result in a probabilistic change of state, for better or worse.
By computing expected value\index{expected value} of every action
in every state, we can compute the optimal decision in every situation.  

%%%%%%%%%%%%%%%%%%%%%%%%%%%%%%%%%%%%%%%%%%%%%%%%%%%%%%%%%%%%%%%%%%%%%%%%%%%%%%%
\section{Dealer's Play}
\label{sec:rules:dealer-play}

\begin{table}[ht!]
\caption{Dealer's hitting state transition table}
\begin{center}
% \begin{savenotes}
\input tables/dealer/dealer-hit-HS17
% \end{savenotes}
\end{center}
\label{tab:dealer-hit-HS17}
\end{table}

\begin{comment}
\begin{table}[ht!]
\caption{Dealer's hitting state transition table (hits on soft-17)}
\begin{center}
\input tables/dealer/dealer-hit-H17
\end{center}
\label{tab:dealer-hit-H17}
\end{table}

\begin{table}[ht!]
\caption{Dealer's hitting state transition table (stands on soft-17)}
\begin{center}
\input tables/dealer/dealer-hit-S17
\end{center}
\label{tab:dealer-hit-S17}
\end{table}
\end{comment}

The dealer's process (to hit until a terminal state is reached)
is entirely pre-determined and summarized by Table~\ref{tab:dealer-hit-HS17}.
Table entry row-$i$ (hand state), column-$j$ (hit card) is the enumeral value
of state \Trans{i}{j}.  For many cases, the new state is just $row+column$.
The blank rows represent the dealer's terminal states --
these rows can be filled with state $i$, representing no state change.
States labeled `A,$x$' represent soft-totals, where the Ace (A)
can be evaluated as 1 or 11.  
The terminal `A,$x$' states are treated as the higher of the 
two possible values, ranging from 17 to 21.  
The table varies in row A,6 between the hit-on-soft-17 (H17)
and stand-on-soft-17 (S17) variations: for S17, A,6 is always treated
as a terminal hard-17.
Later, we will compare the edge impact of H17 vs. the S17 variation.

In reading these tables, 
the first dealer `hit' simply reveals the hole card, 
which was peeked\index{peek} and known not to give the 
dealer a natural blackjack in his first two cards.
Thus, the transition edges for \Trans{A}{10}
and \Trans{10}{A} result in 21, and not a natural blackjack.
Without peeking, these entries would be set to `BJ'.

\begin{comment}
Proper analysis begins by evaluating the probability distribution
of the dealer's terminal states given the dealer's up-card. 
Since the dealer's actions are entirely independent of 
the players' actions, we can define the state-machine
for the dealer's cards, following the rule that the dealer
must hit until his total exceeds 16.
\end{comment}

In Chapter~\ref{sec:dealer}, 
we compute the impact of the dealer's up-card 
on the probability spread of the dealer's final states.

%%%%%%%%%%%%%%%%%%%%%%%%%%%%%%%%%%%%%%%%%%%%%%%%%%%%%%%%%%%%%%%%%%%%%%%%%%%%%%%
\section{Player's State Transitions}
\label{sec:rules:player-hit}

\begin{table}[ht!]
\caption{Player's hitting state transition table}
\begin{center}
\input tables/player-hit
\end{center}
\label{tab:player-hit}
\end{table}

Every `hit' by the player is also represented as a state transition.
Unlike the dealer, however, the player always has the option
to stand in any state.
Table~\ref{tab:player-hit} shows all of the valid changes of state
when taking another card, and is similar to the dealer's table.
This table just shows possible transitions, 
not all of them are necessarily favorable.

\begin{table}[ht!]
\caption{Player's final split state transition table}
\begin{center}
\input tables/player-final-split
\end{center}
\label{tab:player-final-split}
\end{table}

\begin{table}[ht!]
\caption{Player's resplit state transition table}
\begin{center}
\input tables/player-resplit
\end{center}
\label{tab:player-resplit}
\end{table}

For paired hands, the player is given the option to split the
hand into two new hands, with each receiving a new up-card on top
of the original split card.
If the player is only allowed to split to two hands
(this is her last exercisable split), 
then the transition table is Table~\ref{tab:player-final-split}.
If further splitting is permitted, then the state transition table
is Table~\ref{tab:player-resplit}.

