% "composition.tex"

\chapter{Composition-dependent Strategy}
\label{sec:composition}
\index{composition-dependent}

This chapter examines the variations in strategy that result in
removing one of the simplifications used in Chapter~\ref{sec:basic}.
The underlying principle of a composition-dependent strategy (CDS)
is that the exposed cards affect the distribution of the remaining
cards in the deck.
The number of decks is relevant in determining
the sensitivity of distribution to card exposure, 
so the analyses presented in this chapter 
will be repeated for different numbers of decks.

The analyses in this chapter make a few assumptions and approximations:
\begin{itemize}
\item The deck is fresh, i.e. starts with the standard deck distribution before
any cards are dealt.
\item The distribution of cards remaining in the deck is \emph{static} 
after considering the initially exposed cards, i.e. subsequently exposed 
cards (after hitting) do not affect change the distribution.  
\end{itemize}

We introduce a few more conventions used throughout this chapter:
\begin{itemize}
\item Let $\Vec{c^*}$ represent the standard deck card distribution.  
\item Let $\Vec{c}\Set{R}$ represent the distribution of cards
	after cards $R$ have been removed from a single deck.
\item Let $\Vec{Nc}\Set{R}$ represent the distribution of cards
	after cards $R$ have been removed from $N$ decks.
\end{itemize}

%%%%%%%%%%%%%%%%%%%%%%%%%%%%%%%%%%%%%%%%%%%%%%%%%%%%%%%%%%%%%%%%%%%%%%%%%%%%%%%
\section{Single-deck Strategy}
\label{sec:composition:strategy}

Recompute dealer final states given
\begin{itemize}
\item only dealer's revealed card (10 cases)
\item player's cards and dealer's revealed card (900 cases)
\end{itemize}

%%%%%%%%%%%%%%%%%%%%%%%%%%%%%%%%%%%%%%%%%%%%%%%%%%%%%%%%%%%%%%%%%%%%%%%%%%%%%%%
\section{Composition-dependence Effectiveness}
\label{sec:composition:eff}

Discuss the effectiveness of CDS.
1 deck, 2, 4.

