% "basic.tex"

\chapter{Basic Strategy}
\label{sec:basic}

This chapter not only reveals the well-known optimal basic strategy
of Blackjack, but it also mathematically derives the strategy 
in great detail.

Assumptions: infinite deck approximation, as if cards drawn are
immediately replaced to eliminate the effects of history.

%%%%%%%%%%%%%%%%%%%%%%%%%%%%%%%%%%%%%%%%%%%%%%%%%%%%%%%%%%%%%%%%%%%%%%%%%%%%%%%
\section{Hit vs. Stand}
\label{sec:basic:hit-stand}

The most common decision the places faces in any situation
is whether to hit or stand.

%%%%%%%%%%%%%%%%%%%%%%%%%%%%%%%%%%%%%%%%%%%%%%%%%%%%%%%%%%%%%%%%%%%%%%%%%%%%%%%
\section{Double-down}
\label{sec:basic:double}

\index{double-down}
On the player's first two cards, the player is given the option to
double-down: double the original bet and take exactly one more card.

%%%%%%%%%%%%%%%%%%%%%%%%%%%%%%%%%%%%%%%%%%%%%%%%%%%%%%%%%%%%%%%%%%%%%%%%%%%%%%%
\section{Surrender}
\label{sec:basic:surrender}

\index{surrender}
Some blackjack games allow an option to surrender, 
to end the hand by taking back half of the original
bet and giving up half.
There exist situation where the expected outcome of any other action 
is worse than just losing half of the bet outright.  

There exist two types of surrender:
early-surrender\index{surrender!early} and late-surrender{surrender!late}.
Early-surrender is very rare, it allows the player to 
end the hand by losing half of the original bet \emph{before}
the dealer peeks for blackjack.
Late-surrender is more common, and give the player the opportunity to
surrender the hand, again losing half the bet, 
at the time the player is prompted to act on her first two cards, 
which always happens \emph{after} peeking for blackjack.  
With late-surrender, the player still loses her whole bet against 
a dealer natural.
For now, we only evaluate late-surrender, as it is far more common.


%%%%%%%%%%%%%%%%%%%%%%%%%%%%%%%%%%%%%%%%%%%%%%%%%%%%%%%%%%%%%%%%%%%%%%%%%%%%%%%
\section{Split}
\label{sec:basic:split}

The player is always given the opportunity to split
the initial hand when it consists of equal-valued cards.
Any two 10-valued cards are also splittable.

\subsection{Double-after-split}
\label{sec:basic:DAS}

\index{double-after-split}

\subsection{Resplit}
\label{sec:basic:resplit}

\index{resplit}

\subsection{Split Aces}
\label{sec:basic:split-aces}

\index{split aces}
Paired Aces are often given exceptional rules in some games.
Some variations limit the resplitting of Aces, 
and another variation allows no further action after
up-cards are dealt on the split aces.  


%%%%%%%%%%%%%%%%%%%%%%%%%%%%%%%%%%%%%%%%%%%%%%%%%%%%%%%%%%%%%%%%%%%%%%%%%%%%%%%
\section{Insurance}
\label{sec:basic:insurance}

\index{insurance}
In most games, insurance is offered when the dealer reveals an Ace.
Computing the insurance decision is an exercise in probability.
Most insurance payoffs pay 2:1, i.e., the insurance bet (half of the original)
pays the whole bet amount if the dealer does have a natural.
In other words, the insurance bet is equivalent to betting 
whether or not the dealer's hole card is a 10.
In a standard deck, 4 out of 13 cards are 10-valued, 9 out of 13 are not, 
so the odds are 9:4 against.  
If the payoff is only 2:1 (8:4), then insurance is a losing bet.
Hence, basic strategy dictates: \emph{never take insurance}.
Only when the odds of a 10-valued card is less than 2:1 against
is insurance a favorable side bet.

%%%%%%%%%%%%%%%%%%%%%%%%%%%%%%%%%%%%%%%%%%%%%%%%%%%%%%%%%%%%%%%%%%%%%%%%%%%%%%%
\section{Edges}
\label{sec:basic:edges}

% another chapter?
This section quantifies the overall edge of the game.

