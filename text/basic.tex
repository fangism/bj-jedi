% "basic.tex"

\chapter{Basic Strategy}
\label{sec:basic}

This chapter not only reveals the well-known optimal basic strategy
of Blackjack, but it also mathematically derives the strategy 
in full detail.  

Basic strategy is just a summary of all the decisions 
(hit, stand, double-down, split, surrender) that a player can 
make that maximize her expected value.  
Basic strategy can be computed with a few simplifications:
\begin{itemize}
\item Cards are replaced as they are drawn, thus maintaining the same
	probability distribution of cards in the deck.
\item An infinite deck is assumed, so conditional probabilities
	involving the peeked cards are no different from
	the unconditional probabilities.\index{infinite deck}
\item The strategy discussed in this chapter only depends on the 
	total value of each hand, known as
	\emph{total-dependent}\index{total-dependent}, 
	as opposed to \emph{composition-dependent}\index{composition-dependent}.
	In fact, using the same full-deck distribution essentially
	ignores the fact that 2 player cards and 1 dealer card
	are exposed.
	A composition-dependent strategy
	(discussed in Chapter~\ref{sec:composition})
	accounts for the exposed cards.
\item There is no memory of cards previously seen and discarded; 
	the deck is always fresh.
\end{itemize}

\noindent
These simplifications result in quickly computed approximations in the
mathematical edges, that form these basis for decision optimization.  
For increased accuracy at the expense of more computation time,
the above simplifications can be omitted.

Here are some of the mathematical conventions we use
throughout this document.

\begin{itemize}
\item The value of a player's hand is represented in the state $s_i$.
\item Let $\NextState{s}{i}{j}$ denote the state of the hand
when the hit card $j$ is added to state $s_i$.
\item Let $\Vec{c}=\Angle{c_1, \ldots, c_{10}}$ be the probability 10-vector
	that represents the card probability distribution.
\item Let $\ExpAct{a}{s_i}$ denote the expected value of taking action
$a$ (enumerated $hit$, $stand$, $double$, $split$, $surrender$) in state $s_i$.
\item Let $\ExpOpt{s_i}$ denote the optimal expected value in state $s_i$.
\item Let $\Vec{s_i}$ denote the vector of states reached by adding 
	one card to state $s_i$, that is, $\Angle{\Loop{,}{j}{10} \NextState{s}{i}{j}}$
\item Let $\ExpVec{s_i}$ denote the vector of expected values of 
	states reached by adding one card, that is, 
	$\Angle{\Loop{,}{j}{10} \Exp{\NextState{s}{i}{j}}}$
\end{itemize}

%%%%%%%%%%%%%%%%%%%%%%%%%%%%%%%%%%%%%%%%%%%%%%%%%%%%%%%%%%%%%%%%%%%%%%%%%%%%%%%
\section{Hit vs. Stand}
\label{sec:basic:hit-stand}

The most common decision the player faces in any situation
is whether to hit or stand.
It is also the most important to learn and master first.

\subsection{Standing}
\label{sec:basic:hit-stand:stand}

\begin{table}[ht!]
\caption{Player's stand edges (H17, pre-peek)}
\begin{center}
\input tables/player-stand-H17-pre-peek
\end{center}
\label{tab:player-stand-H17-pre-peek}
\end{table}

\begin{table}[ht!]
\caption{Player's stand edges (H17, post-peek)}
\begin{center}
\input tables/player-stand-H17-post-peek
\end{center}
\label{tab:player-stand-H17-post-peek}
\end{table}

% matrix
Let us interpret Table~\ref{tab:win-lose} as a payoff matrix $R$
(row $p$ for player state, column $d$ for dealer state)
whose cell values $R_{p,d}$ are:
$+1$ for a win, $-1$ for lose, and $0$ for push, 
and $+1.5$ (or whatever the blackjack payoff is) for win*.

If we combine Table~\ref{tab:win-lose} and
Table~\ref{tab:dealer-final-H17} (H17, pre-peek) 
the result is the expected winnings per unit bet, 
is Table~\ref{tab:player-stand-H17-pre-peek}.

For post-peek conditions, Table~\ref{tab:win-lose} is combined with
Table~\ref{tab:dealer-final-H17} (H17, post-peek) to produce
Table~\ref{tab:player-stand-H17-post-peek}.
The only columns which differ between these tables are when
the dealer's reveal card is an Ace or 10, since those are the only cards
that provide information about the hole card
when the dealer peeks for blackjack.
Under most variations of Blackjack where the player is prompted for
action after the dealer peeks for a natural,
the strategy for optimizing decisions should be based
on post-peek conditions.  

Intuitively, as the player's hand value increases up to 21, 
her expected outcome also monotonically increases.
This is trend is expected because the corresponding player value rows in
Table~\ref{tab:win-lose} show the sets of winning scenarios
grow with hand value, and the set of losing scenarios diminish with
hand value.  

\begin{table}[ht!]
\caption{Player's stand edges (S17, pre-peek)}
\begin{center}
\input tables/player-stand-S17-pre-peek
\end{center}
\label{tab:player-stand-S17-pre-peek}
\end{table}

\begin{table}[ht!]
\caption{Player's stand edges (S17, post-peek)}
\begin{center}
\input tables/player-stand-S17-post-peek
\end{center}
\label{tab:player-stand-S17-post-peek}
\end{table}

For the S17 variation, the same computation produces
Table~\ref{tab:player-stand-S17-pre-peek} (pre-peek), and
Table~\ref{tab:player-stand-S17-post-peek} (post-peek).

Decision evaluations will use the post-peek tables when
comparing to against the expected outcomes of other actions, 
because the player does not have a chance to act before
being assured that the dealer does not have blackjack.  
% exception: no peek variation

\subsection{Hitting}
\label{sec:basic:hit-stand:hit}

The obvious hit decisions are when the player's hand is 11 or lower,
as hitting cannot result in a bust, and since any total $\le{}16$ is
worth the same, the hand can only improve by hitting.  
% If it is better to stand than hit at X, then it is also better to
% stand than hit at all X+k (for hard totals).

The expected outcome of hitting in each state $s_i$ 
can be computed recursively:

\begin{eqnarray}
\ExpOpt{s_i} &=& \Max{\ExpAct{hit}{s_i},\ExpAct{stand}{s_i}} \label{eqn:opt:hit-stand}\\
\ExpAct{hit}{s_i} &=& \Sum{j=1}{10}{\Prob{c=j}\ExpOpt{\NextState{s}{i}{j}}}
\end{eqnarray}

\noindent
To follow dependency ordering, the expectations $\ExpAct{hit}{s_i}$
must be evaluated starting from the player's terminal states 
(21, bust) and working backwards.  

\begin{center}
\begin{longtable}{|c||c|c|c|c|c||c|c|c|c|c|}
\caption{Player hit vs. stand expectations (H17)%
\label{tab:hit-stand-expectations-H17}}\\ \hline
\tblULhdr{reveal}{player}
&2&3&4&5&6&7&8&9&10&A \\ \hline \hline
\endfirsthead
\caption[]{(continued)}\\ \hline \hline
\tblULhdr{reveal}{player}
&2&3&4&5&6&7&8&9&10&A \\ \hline \hline
\endhead
\small
\input tables/hit-stand-basic-H17
\end{longtable}
\end{center}

\begin{center}
\begin{longtable}{|c||c|c|c|c|c||c|c|c|c|c|}
\caption{Player hit vs. stand expectations (S17)%
\label{tab:hit-stand-expectations-S17}}\\ \hline
\tblULhdr{reveal}{player}
&2&3&4&5&6&7&8&9&10&A \\ \hline \hline
\endfirsthead
\caption[]{(continued)}\\ \hline \hline
\tblULhdr{reveal}{player}
&2&3&4&5&6&7&8&9&10&A \\ \hline \hline
\endhead
\small
\input tables/hit-stand-basic-S17
\end{longtable}
\end{center}

%%%%%%%%%%%%%%%%%%%%%%%%%%%%%%%%%%%%%%%%%%%%%%%%%%%%%%%%%%%%%%%%%%%%%%%%%%%%%%%
\section{Double-down}
\label{sec:basic:double}

\index{double-down}
On the player's first two cards, the player is given the option to
double-down: double the original bet and take exactly one more card.
The expected value of doubling-down is computed as an inner-product
of the card distribution vector and the vector of expected values for the
post-double-down states, multiplied by the double-down multiplier, $d$.
Typically, the cost of doubling-down is the original bet amount, 
so $d$ is 2.0.

\begin{eqnarray}
\ExpAct{double}{s_i} &=& d \Paren{\Vec{c}\cdot \ExpActVec{stand}{s_i}} \nonumber \\
&=& d \Sum{j=1}{10}{\Prob{c=j}\ExpAct{stand}{\NextState{s}{i}{j}}}
\end{eqnarray}

Where doubling-down is permitted, the expression for the
optimal expected value (Equation~\ref{eqn:opt:hit-stand}) gains one more term:
\begin{eqnarray}
\ExpOpt{s_i} &=& \Max{\ExpAct{hit}{s_i},\ExpAct{stand}{s_i},\ExpAct{double}{s_i}} \label{eqn:opt:hit-stand-dbl}
\end{eqnarray}

%%%%%%%%%%%%%%%%%%%%%%%%%%%%%%%%%%%%%%%%%%%%%%%%%%%%%%%%%%%%%%%%%%%%%%%%%%%%%%%
\section{Surrender}
\label{sec:basic:surrender}

\index{surrender}
Some blackjack games allow an option to surrender, 
to end the hand by taking back half of the original
bet and giving up half.
There exist situation where the expected outcome of any other action 
is worse than just losing half of the bet outright.  

There exist two types of surrender:
early-surrender\index{surrender!early} and late-surrender{surrender!late}.
Early-surrender is very rare, it allows the player to 
end the hand by losing half of the original bet \emph{before}
the dealer peeks for blackjack.
Late-surrender is more common, and give the player the opportunity to
surrender the hand, again losing half the bet, 
at the time the player is prompted to act on her first two cards, 
which always happens \emph{after} peeking for blackjack.  
With late-surrender, the player still loses her whole bet against 
a dealer natural.

\subsection{Late Surrender (common)}
\label{sec:basic:surrender:late}
To evaluate late-surrender, one simply compares the expected outcome
of any action (under post-peek conditions) against an outcome of $-0.5$.
When all other actions have a worse expected outcome than the surrender
penalty, it is to the player's advantage to surrender.

\subsection{Early Surrender (rare)}
\label{sec:basic:surrender:early}
Since early surrender is rare, the reader may wish to skip this section.
Early surrender (when allowed) gives the player the chance
to surrender the hand before the dealer peeks at the hole card
for a blackjack, when a 10 or Ace is showing.
To determine early surrender when is favorable, 
one must examine the \emph{pre-peek} expectations.

\TODO: add columns to hit/stand tables vs. unpeeked 10 and A

%%%%%%%%%%%%%%%%%%%%%%%%%%%%%%%%%%%%%%%%%%%%%%%%%%%%%%%%%%%%%%%%%%%%%%%%%%%%%%%
\section{Split}
\label{sec:basic:split}

The player is always given the opportunity to split
the initial hand when it consists of equal-valued cards.
Any two 10-valued cards are also splittable under most blackjack variants.

\subsection{Double-after-split}
\label{sec:basic:DAS}

\index{double-after-split}

\subsection{Resplit}
\label{sec:basic:resplit}

\index{resplit}

\subsection{Split Aces}
\label{sec:basic:split-aces}

\index{split aces}
Paired Aces are often given exceptional rules in some games.
Some variations limit the resplitting of Aces, 
and another variation allows no further action after
up-cards are dealt on the split aces.  

\subsection{Expected Value of Splitting Hands}
\label{sec:basic:split:calc}

Precise calculations of expected value of splitting hands can be 
compute-intensive, especially when exhausting the number of 
possible hands and states after splitting and resplitting.
Methods exist for accelerating exact calculations by 
reusing sub-solutions in the recursively expanded 
computation graph~\cite{ref:nairn08}.
In practice, approximations of the expected value are adequate 
substitutes for exact values~\cite{ref:griffin99}.  
This section presents an improved method for calculating exact 
expected values of splitting hands.

Consider an initially paired hand $XX$ (composed of two $X$ cards)
with splitting allowed up to $S+1$ hands.
(We let $S$ be the number of \emph{times} a player may split, 
so with a maximum of 4 hands, $S=3$.)
If the player decides to split, the resulting state is
two new hands $XY$ and $XZ$ with further splitting allowed up to $S$ hands 
(one less than before).  
Depending on the values of $Y$ and $Z$, the new state may include 
0 through 2 splittable hands (one less to one more than before).  
By symmetry, the order in which hands $XY$ and $XZ$ are played 
do not affect the overall expected value.  

Let us consider the cases where new cards
$Y$ and $Z$ form 0, 1, or 2 newly splittable hands, 
without concern for the values of the new cards on the non-splittable hands.
If any splittable hands remain and the number of hands has not exceeded
the limit, the player may elect to split further.
Let us further restrict the players decision in each state to 
split or not split.
At each step of this simplified process, 
the state of all yet-to-be-acted-upon hands can be 
summarized with a few parameters:

\begin{itemize}
\item $S$ is the number of allowed splits remaining
\item $N$ is the number of non-paired hands in play
\item $P$ is the number of paired hands $XX$ in play, which consists of:
\begin{itemize}
\item $P_p=\Min{P,S}$, the number of paired hands not exceeding $S$
\item $P_n=P-P_p$, the number of paired hands in excess of $S$, 
	and are treated as non-splittable
\end{itemize}
\end{itemize}

By definition, the following constraint must hold:
\begin{eqnarray}
N+P \le S+1 \nonumber \\
N+P_p+P_n \le S+1
\end{eqnarray}

\noindent
Let $p_X=\Prob{c=X}$ be the probability of drawing card $X$ to form a 
newly paired $XX$ hand, and $q_X=1-p_X$.
Let $P^\prime$ denote the new number of paired hands after 
splitting one of the $P$ paired hands, which can range from $P-1$ to $P+1$.
The outcome probability distribution of $P^\prime$ can be approximated 
by assuming an infinite deck (same as drawing without replacement):
\begin{eqnarray}
\Prob{P^\prime=P-1} &=& q_X^2 \\
\Prob{P^\prime=P} &=& 2 p_X q_X \\
\Prob{P^\prime=P+1} &=& p_X^2
\end{eqnarray}

\noindent
We can express the expected value in each state
with an arbitrary number of paired $XX$ hands, recursively.
For $S \ge 1$ and $P \ge 1$:

\begin{eqnarray}
\ExpAct{split}{XX,S,P,N} &=& \Prob{P^\prime=P-1}\ExpOpt{XX,S-1,P-1,N+2} \nonumber \\
&+& \Prob{P^\prime=P}\ExpOpt{XX,S-1,P,N+1} \nonumber \\
&+& \Prob{P^\prime=P+1}\ExpOpt{XX,S-1,P+1,N} \label{eqn:split:EV:recursion} \\
\ExpOpt{XX,S,P,N} &=& \Max{\ExpOpt{XX},\ExpAct{split}{XX,S,P,N}} \\
\ExpOpt{XX} &=& \Max{\ExpAct{hit}{XX},\ExpAct{stand}{XX},\ExpAct{double}{XX}}
\end{eqnarray}

\noindent
$\ExpOpt{XX}$ is just the optimal action in a single hand in state $XX$
without the option to split.

When $S=0$ (no more splits are allowed), or $P=0$ (no more paired hands):
\begin{eqnarray}
\ExpAct{split}{XX,0,P,N} &=& P_n\cdot\ExpOpt{XX} +N\cdot\ExpOpt{XY}\\
\ExpOpt{XY} &=& \Sum{j}{10}{\Prob{c=j|c\neq{}X} \ExpOpt{\Trans{X}{j}}}
\end{eqnarray}

In the evaluation of the expected value of splitting, 
Equation~\ref{eqn:split:EV:recursion} is recursively expanded until
no terms contain $S>0$.

We illustrate the recursive expansion of the expected value expressions 
with a more concrete example, 
using the following shorthand notation that represents the 
set of possible split hand configurations and their probabilities.

\begin{itemize}
\item $P$ is a hand that is paired and may be split
\item $X$ is a paired hand that may not be split
\item $Y$ is a non-paired hand
\item $p$ is the probability of drawing the paired card
\item $q$ is $1-p$
\item $\Brack{\ldots}$ is a state with a string of
$P$, $X$, and $Y$ hands, e.g. $\Brack{PPXYY}$.
As a convention, we sort $P$s before $X$s, $X$s before $Y$s.
\item any leading coefficient is the repetition count for a state
\item Let $M_P$, $M_X$, and $M_Y$ be the number of $P$, $X$, and $Y$
hands in a string, respectively.  
\end{itemize}

\begin{figure}[ht!]
\centerline{\resizex{2in}{\xfig{figs/split-graph-2}}}
\caption{Evaluation graph for splitting allowed up to 2 hands}
\label{fig:split-2}
\end{figure}

\begin{figure}[ht!]
\centerline{\resizex{3in}{\xfig{figs/split-graph-3}}}
\caption{Evaluation graph for splitting allowed up to 3 hands}
\label{fig:split-3}
\end{figure}

\begin{figure}[ht!]
\centerline{\resizex{4in}{\xfig{figs/split-graph-4}}}
\caption{Evaluation graph for splitting allowed up to 4 hands}
\label{fig:split-4}
\end{figure}

\noindent
The expected value of splitting in a given state
is just the probability weighted sum of 
the maximum expected values of the post-split states. 
Figures~\ref{fig:split-2}, \ref{fig:split-3}, \ref{fig:split-4}
show the evaluation graphs
for splitting up to 2, 3, and 4 hands, respectively.  
The initial state in each graph is a single paired hand, $P$.  
The $N$ in $N:\ldots$ represents the number of splits remaining.
Red edges are transitions with probability $p^2$,
green edges have probability $2pq$,
and blue edges have probability $q^2$.

Note that these evaluation graphs may contain reconvergent paths
to the same state; there is more than one way to reach some states.
Computed purely recursively, the same state (and all of its descendants)
would be evaluated multiple times, 
often resulting in an exponentially growing amount of computation.  
Conventional techniques for accelerating such computations
include memoization and caching of intermediate results.  

% In Figures~\ref{fig:split-2} through~\ref{fig:split-4}, 
% the nodes are also annotated with a number $(M)$, 
% that represents the ...
\TODO: count paths to terminal nodes before and after caching,
compare number of node evaluations
% write generic dot-graph reader to count paths from root to terminals?

\begin{comment}
For $S=1,P=1$:
\begin{eqnarray}
P &\goesto& p^2 \Brack{XX} +2pq \Brack{XY} +q^2 \Brack{YY}
\end{eqnarray}

For $S=2,P=1$:
\begin{eqnarray}
P &\goesto& p^2\Brack{PX} +2pq\Brack{PY} +q^2\Brack{YY} \\
&\goesto& p^2\Brack{X\Paren{p^2\Brack{XX} +2pq\Brack{XY} +q^2\Brack{YY}}}
 +2pq\Brack{Y\Paren{p^2\Brack{XX} +2pq\Brack{XY} +q^2\Brack{YY}}}
 +q^2\Brack{YY} \\
&\equiv& \ldots
\end{eqnarray}

For $S=3,P=1$:
\begin{eqnarray}
P &\goesto& PP +2PY +YY \\
&\goesto& \ldots
\end{eqnarray}
\end{comment}

% subsubsection: inaccuracies and approximations
It is important to note that above method is still an approximation,
due to the fact that:

\begin{enumerate}
\item We have not accounted for the effect of card removal as 
each split hand receives new cards.
\item Enumerating split states with hand counts $(M_P, M_X, M_Y)$ 
ignores permutation, 
and disregards the \emph{sequence} in which the set of live hands is played. 
\end{enumerate}

\noindent
With an infinite deck, the exposure of cards has no effect on the
card distribution for subsequent hands, therefore all hands can be 
treated \emph{independently}.  
The effect of card removal will be addressed in Chapter~\ref{sec:counting}.


% % % % % % % % % % % % % % % % % % % % % % % % % % % % % % % % % % % % % % % %
\subsection{Action Expectations}
\label{sec:basic:action-expectations}

% TODO: smaller font
\begin{center}
\begin{longtable}{|c||c|c|c|c|c||c|c|c|c|c|}
\caption{Player's action expectations (H17)%
\label{tab:player-action-expectations-H17}}\\ \hline
\tblULhdr{reveal}{player}
&2&3&4&5&6&7&8&9&10&A \\ \hline \hline
\endfirsthead
\caption[]{(continued)}\\ \hline \hline
\tblULhdr{reveal}{player}
&2&3&4&5&6&7&8&9&10&A \\ \hline \hline
\endhead
\small
\input tables/player-action-expectations-basic-H17
\end{longtable}
\end{center}

\begin{center}
\begin{longtable}{|c||c|c|c|c|c||c|c|c|c|c|}
\caption{Player's action expectations (S17)%
\label{tab:player-action-expectations-S17}}\\ \hline
\tblULhdr{reveal}{player}
&2&3&4&5&6&7&8&9&10&A \\ \hline \hline
\endfirsthead
\caption[]{(continued)}\\ \hline \hline
\tblULhdr{reveal}{player}
&2&3&4&5&6&7&8&9&10&A \\ \hline \hline
\endhead
\small
\input tables/player-action-expectations-basic-S17
\end{longtable}
\end{center}

Table~\ref{tab:player-action-expectations-H17} summarizes of the
expectations per player action in every situation against the dealer
for the H17 variation.
Table~\ref{tab:player-action-expectations-S17} summarizes of the
expectations per player action in every situation against the dealer
for the S17 variation.

%%%%%%%%%%%%%%%%%%%%%%%%%%%%%%%%%%%%%%%%%%%%%%%%%%%%%%%%%%%%%%%%%%%%%%%%%%%%%%%
\section{Insurance}
\label{sec:basic:insurance}

\index{insurance}
In most games, insurance is offered when the dealer reveals an Ace.
Computing the insurance decision is an exercise in probability.
Most insurance payoffs pay 2:1, i.e., the insurance bet (half of the original)
pays the whole bet amount if the dealer does have a natural.
In other words, the insurance bet is equivalent to betting 
whether or not the dealer's hole card is a 10.
In a standard deck, 4/13 cards are 10-valued, 9/13 are not, 
so the odds are 9:4 against.  
If the payoff is only 2:1 (8:4), then insurance is a losing bet.
Hence, basic strategy dictates: \emph{never take insurance}.
Only when the odds of a 10-valued card is less than 2:1 against
is insurance a favorable side bet.

\subsection{Even-money}
\label{sec:basic:insurance:even-money}

\index{even money}
When the player has a natural blackjack, and the dealer is showing an 
Ace, the dealer sometimes offers the ``even money'' proposition, 
before peeking for a 10 in the hole.
If the player accepts, then she wins 1:1 on her original bet
instead of the typical 3:2, and the hand is over.
If the player declines, then the dealer peeks for a 10.
If the hole card is a 10, giving the dealer a natural, then the hand
is declared a push, and no money is exchanged.
Otherwise, the player's natural holds up against the dealer
and is paid out 3:2.

Under these circumstances, with the standard 3:2 blackjack payoff, 
even money is equivalent to the insurance side bet;
accepting ``even money'' is to the player's disadvantage.
With the standard card distribution, 
declining even money pays off 3:2 with probability 9/13, 
while the other 4/13 times pays nothing (push), 
so the expected value of declining is
$\frac{9}{13}\frac{3}{2}=\frac{27}{26}$,
whereas the expected value of accepting even money is only 1.  
Since $\frac{27}{26}>1$, it is better to decline even money, 
under basic strategy.

\begin{exercise}
Analyze the even money proposition
given a 6:5 payoff for natural blackjack.
\end{exercise}

%%%%%%%%%%%%%%%%%%%%%%%%%%%%%%%%%%%%%%%%%%%%%%%%%%%%%%%%%%%%%%%%%%%%%%%%%%%%%%%
\section{Peeking}
\label{sec:basic:peeking}

\index{peeking}
This section addresses the impact of the peeking for a natural blackjack
on the odds of drawing cards, and computing probabilities of outcomes.
If you are only interested in strategy, you may skip this section.  

Under common blackjack variants, the dealer peeks at the hole card
to see if he already has a natural blackjack.
The dealer can only have a natural if the dealer's reveal card is an 
Ace or 10-valued, otherwise there is no need to peek at the hole card.
When an Ace is showing, the dealer checks for a 10 in the hole (usually 
after offering the insurance side bet), and when a 10 is showing, 
the dealer checks for an A in the hole (usually without side bets).
When the dealer does have a natural, the hand is over, 
and there is nothing else to calculate.
If the dealer does not have a natural, then play resumes, 
and the player is next to act, with additional information on the
distribution of cards remaining in the deck.

\subsection{Bayes' Theorem}
\label{sec:basic:peeking:bayes}
\index{Bayes' Theorem}
\index{conditional probability}
\index{probability!conditional}

Bayes' Theorem defines a mathematical relationship between
probabilities of two events and the \emph{conditional probabilities}
of one event given the other.

\begin{eqnarray}
\Prob{A|B} &=& \frac{\Prob{B|A}\Prob{A}}{\Prob{B}} \label{eqn:bayes}
\end{eqnarray}

Bayes' Theorem applies to computing post-peek conditional probabilities
because knowledge of one event (eliminated hole card value)
influences the probability distribution of drawing the next card.

We define events $A$ and $B$ prior to the peeking of any hole cards.
\begin{itemize}
\item Let $A_m$ be the event of drawing card $m:m\neq{}10$ 
from the deck of $N$ cards.
\item Let $A_{10}$ be the event of drawing a 10 card from the same deck.
\item Let $B$ be the event of drawing a non-10 card from the same deck.
\end{itemize}

\noindent
Let $N_k$ denote the number of $k$-valued cards in the deck of size $N$,
with $\Sum{k=1}{10}=N$.

\noindent
Let $N_{-k}=N-N_k$, for shorthand.

\noindent
Their respective probabilities are:
\begin{eqnarray}
\Prob{A_m} &=& \frac{N_m}{N}\\
\Prob{A_{10}} &=& \frac{N_{10}}{N}\\
\Prob{B} &=& \frac{N_{-10}}{N}
\end{eqnarray}

\noindent
The conditional probability of $B|A_m$ follows from removing a non-10 card
from the deck, 
and the conditional probability of $B|A_{10}$ follows from removing a 10 card
from the deck.

\begin{eqnarray}
\Prob{B|A_m} &=& \frac{N-1-N_{10}}{N-1} \nonumber \\
&=& \frac{N_{-10}-1}{N-1}\\
\Prob{B|A_{10}} &=& \frac{N-1-\Paren{N_{10}-1}}{N-1} \nonumber \\
&=& \frac{N_{-10}}{N-1}
\end{eqnarray}

\noindent
It follows from Bayes' Theorem (Equation~\ref{eqn:bayes}) that
the conditional probabilities of $A|B$ are:

\begin{eqnarray}
\Prob{A_m|B} &=& \frac{N_m}{N}\frac{N_{-10}-1}{N-1}\frac{N}{N_{-10}} \nonumber \\
&=& \frac{N_m}{N-1}\frac{N_{-10}-1}{N_{-10}} \label{eqn:Am_B} \\
\Prob{A_{10}|B} &=& \frac{N_{10}}{N}\frac{N_{-10}}{N-1}\frac{N}{N{-10}-1} \nonumber \\
&=& \frac{N_{10}}{N-1}\frac{N_{-10}}{N{-10}-1} \label{eqn:A10_B}
\end{eqnarray}

Relative to the \emph{priori} probabilities $\Prob{A_m}$ the
\emph{posteriori} probabilities makes sense;
eliminating a non-10 card from the deck increases the likelihood
of drawing a 10 card next, and decreases the likelihood of drawing
a non-10 card next.
The same analysis applies to the situation where the hole card
is known to not be an Ace.

As a sanity check, we confirm that $\Sum{j=1}{10}{\Prob{A_j|B}}$ = 1.
\begin{eqnarray}
\Sum{j=1}{10}{\Prob{A_j|B}} &=&
\Prob{A_{10}|B} +\Sum{j}{j\neq{}10}{\Prob{A_j|B}}\\
&=& \frac{N_{10}}{N-1}
  +\frac{N_{-10}-1}{\Paren{N-1}\Paren{N_{-10}}} \Sum{j}{j\neq{}10}{N_j} \label{eqn:post_peek} \\
&=& \frac{N_{10}}{N-1} +\frac{\Paren{N_{-10}}\Paren{N_{-10}-1}}{\Paren{N-1}\Paren{N_{-10}}}\\
&=& \frac{N_{10}}{N-1} +\frac{N_{-10}-1}{N-1}\\
&=& \frac{N_{10}+N_{-10}-1}{N-1}\\
&=& 1
\end{eqnarray}

Note also that in the infinite-deck limit, there is no difference
between the \emph{priori} and \emph{posteriori} probabilities because
the knowledge of the hole card no longer affects the relative distribution of 
remaining cards:
\begin{eqnarray}
\forall k: \LimInf{N}{\Prob{A_k|B}} &=& \frac{N_k}{N}
\end{eqnarray}

\noindent
Example calculations:
\begin{center}
\begin{tabular}{|c|c|c||c|c|}
\hline
$N$ & $N_{10}$ & $N_m$ & $\Prob{A_{10}|B}$ & $\Prob{A_m|B}$ \\ \hline \hline
13 & 4 & 1 & 4/12 & 2/27 \\ \hline
26 & 8 & 2 & 8/25 & 17/225 \\ \hline
52 & 16 & 4 & 16/51 & 35/459 \\ \hline
\end{tabular}
\end{center}

% table?
Another way to interpret the post-peeking probabilities is by treating 
the term coefficients as weights of the \emph{priori} distribution.
Before peeking, the weight of drawing each card $m$ is exactly
$N_m$, thus each \emph{priori} probability is $N_m/N$.
If we take Equation~\ref{eqn:post_peek} and multiply both sides
by the common denominator to acquire integer weights (whose sum is 
$\Paren{N-1}\Paren{N_{-10}}$), we find the \emph{posteriori} weights:

\begin{eqnarray}
\Prob{A_m|B} &\propto& N_m \Paren{N_{-10}-1} \\
\Prob{A_{10}|B} &\propto& N_{10} \Paren{N_{-10}}
\end{eqnarray}

This unequal weighting accounts for the relative change in 
conditional probabilities given event $B$.
Eliminating the common denominator can be used to efficiently implement the
random card generator without having to perform division.

\subsection{Accumulating unknown hole cards}
\label{sec:basic:peeking:accumulate}

As a deck or shoe is played, there may be times when the hole card
is never revealed.  This can occur if the player (or all players)
bust or surrender. 
When this happens, the dealer need not play out his hand; 
he may discard his revealed card and the hole card without revealing
the hole card.
When the dealer shows neither a 10 nor Ace, 
there is no information about the hole card;
it may as well be replaced among the remaining cards.  
However, when the dealer peeks for a natural blackjack,
there is partial information about the hole card
and the probability distribution of the remaining cards.  
When it is the dealer's turn to play, 
revealing the hole card provides certain information about 
what cards remain in the deck.  
The previous section assumed that prior to the hand, 
no peeked hole cards were discarded in previous hands.
This section computes the impact of accumulating discarded peeked hole cards
on probability of drawing cards.  

The previous problem can be generalized:
What are the \emph{posteriori} probabilities of $A|B^d$
when $B^d$ is the event that $d$ hole cards are known to be not 10?

Event $B^d$ is now the case where $d$ non-10 cards are
removed from a deck of $N$.
The probability of $B^d$ is:

\begin{eqnarray}
\Prob{B^d} &=& \frac{N_{-10}}{N}\frac{N_{-10}}{N-1}\cdots\frac{N_{-10}}{N-\Paren{d-1}} \nonumber \\
&=& \frac{N_{-10}^d}{\Prod{i=0}{d-1}{\Paren{N-i}}}
\end{eqnarray}

\noindent
The conditional probabilities of $B^d|A_k$ follows from 
removing a 10 or non-10 card from the deck.
Bear in mind that as each successive non-10 is drawn, 
the denominator diminishes.

\begin{eqnarray}
\Prob{B^d|A_m} &=& \frac{N_{-10}-1}{N-1}\frac{N_{-10}-1}{N-2}\cdots\frac{N_{-10}-1}{N-d} \nonumber \\
&=& \frac{\Paren{N_{-10}-1}^d}{\Prod{i=1}{d}{\Paren{N-i}}}\\
\Prob{B^d|A_{10}} &=& \frac{N_{-10}}{N-1}\frac{N_{-10}}{N-2}\cdots\frac{N_{-10}}{N-d} \nonumber \\
&=& \frac{N_{-10}^d}{\Prod{i=1}{d}{\Paren{N-i}}}
\end{eqnarray}

\noindent
Applying Bayes' Theorem yields $\Prob{A_k|B^d}$:
\begin{eqnarray}
\Prob{A_m|B^d} &=& \Prob{B^d|A_m}\frac{\Prob{A_m}}{\Prob{B^d}} \nonumber \\
&=& \frac{\Paren{N_{-10}-1}^d}{\Prod{i=1}{d}{\Paren{N-i}}}
\frac{N_m}{N} \frac{\Prod{i=0}{d-1}{\Paren{N-i}}}{N_{-10}^d} \nonumber \\
&=& \frac{\Paren{N_{-10}-1}^d}{\Paren{N-d}}
\frac{N_m}{N} \frac{N}{N_{-10}^d} \nonumber \\
&=& \frac{\Paren{N_{-10}-1}^d}{\Paren{N-d}}
\frac{N_m}{N} \frac{N}{N_{-10}^d} \nonumber \\
&=& \frac{N_m}{N-d} \Paren{\frac{N_{-10}-1}{N_{-10}}}^d \label{eqn:Am_Bd} \\
\Prob{A_{10}|B^d} &=& \Prob{B^d|A_{10}}\frac{\Prob{A_{10}}}{\Prob{B^d}} \nonumber \\
&=& \frac{\Paren{N_{-10}}^d}{\Prod{i=1}{d}{\Paren{N-i}}}
\frac{N_{10}}{N} \frac{\Prod{i=0}{d-1}{\Paren{N-i}}}{N_{-10}^d} \nonumber \\
&=& \frac{\Paren{N_{-10}}^d}{\Paren{N-d}}
\frac{N_{10}}{N} \frac{N}{N_{-10}^d} \nonumber \\
&=& \frac{N_{10}}{N-d} \label{eqn:A10_Bd}
\end{eqnarray}

\noindent
Substituting for $d=1$ gives us Equations~\ref{eqn:Am_B} and~\ref{eqn:A10_B}.

\begin{exercise}
Verify that:
\begin{eqnarray}
\Sum{k}{10}{\Prob{A_k|B^d}} &=& 1
\end{eqnarray}
\end{exercise}

If we consider the relative \emph{posteriori} weights $A_k|B^d$, 
eliminating the common denominator resulst in:

\begin{eqnarray}
\Prob{A_m|B^d} &\propto& N_m \Paren{N_{-10}-1}^d \\
\Prob{A_{10}|B^d} &\propto& N_{10} \Paren{N_{-10}}^d
\end{eqnarray}

Again, the above derivation also works for evaluating
when there are $e$ non-Ace cards discarded,
which we denote as $C^e$.

Now suppose that there are $d$ discards that are known to not be 10,
\emph{and} $e$ discards that are known not to be Ace, 
what is $\Prob{A_k|B^d C^e}$?

\begin{eqnarray}
\Prob{B^d C^e} &=& \frac{N_{10}^d N_A^e}{N \Paren{N-1} \cdots \Paren{N-\Paren{d+e-1}}} \nCr{d+e}{d} \nonumber \\
&=& \frac{N_{10}^d N_A^e}{\Prod{i=0}{d+e-1}{\Paren{N-i}}} \nCr{d+e}{d}
\end{eqnarray}

\noindent
The $\nCr{d+e}{d}$ coefficient comes from binomial multiplicity.

\noindent
The conditional probabilities given $A_m$ ($m\neq{}10,m\neq{}A$), 
$A_{10}$, and $A_A$ are:

\begin{eqnarray}
\Prob{B^d C^e|A_m} &=& \nCr{d+e}{d} \frac{\Paren{N_{-10}-1}^d \Paren{N_{-A}-1}^e}{\Prod{i=1}{d+e}{\Paren{N-i}}}\\
\Prob{B^d C^e|A_{10}} &=& \nCr{d+e}{d} \frac{\Paren{N_{-10}}^d \Paren{N_{-A}-1}^e}{\Prod{i=1}{d+e}{\Paren{N-i}}}\\
\Prob{B^d C^e|A_A} &=& \nCr{d+e}{d} \frac{\Paren{N_{-10}-1}^d \Paren{N_{-A}}^e}{\Prod{i=1}{d+e}{\Paren{N-i}}}
\end{eqnarray}

\noindent
Applying Bayes' Theorem (and simplification) 
yields the conditional probabilities:
\begin{eqnarray}
\Prob{A_m|B^c D^e} &=& \frac{N_m}{N-\Paren{d+e}}
\Paren{\frac{N_{-10}-1}{N_{-10}}}^d
\Paren{\frac{N_{-A}-1}{N_{-A}}}^e\\
\Prob{A_{10}|B^c D^e} &=& \frac{N_{10}}{N-\Paren{d+e}}
\Paren{\frac{N_{-A}-1}{N_{-A}}}^e\\
\Prob{A_A|B^c D^e} &=& \frac{N_A}{N-\Paren{d+e}}
\Paren{\frac{N_{-10}-1}{N_{-10}}}^d
\end{eqnarray}

In terms of relative weights, we can multiply each conditional probability
by the common denominator, $\Paren{N-\Paren{d+e}}N_{-10}^d N_{-A}^e$:

\begin{eqnarray}
\Prob{A_m|B^c D^e} &\propto& N_m \Paren{N_{-10}-1}^d \Paren{N_{-A}-1}^e\\
\Prob{A_{10}|B^c D^e} &\propto& N_{10} \Paren{N_{-10}}^d \Paren{N_{-A}-1}^e\\
\Prob{A_A|B^c D^e} &\propto& N_A \Paren{N_{-10}-1}^d \Paren{N_{-A}}^e
\end{eqnarray}

\noindent
With these weight equations, 
one can implement a weighted random number generator with 
\emph{posterior} conditional probabilities, given the number
of peeked discards.  

These \emph{posteriori} weights can be generalized for any number of
peeked cards whose values are eliminated, but for blackjack, 
the equation for peeked 10s and Aces suffices.  

%%%%%%%%%%%%%%%%%%%%%%%%%%%%%%%%%%%%%%%%%%%%%%%%%%%%%%%%%%%%%%%%%%%%%%%%%%%%%%%
\section{Edges}
\label{sec:basic:edges}

% another chapter?
This section quantifies the overall edge of the game.

