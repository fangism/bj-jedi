% "preface.tex"

\subsection*{Preface}
\addcontentsline{toc}{chapter}{Preface}

Even today, I am still averse to gambling and investing.
I've never considered myself lucky at games of chance, 
just ask my old friends with whom I use to play role-playing games
and other roll-of-the-dice board games.  
My limited understanding of competitive games such as 
Chess, Bridge, and Poker
gives me a great appreciation for the masters of those games
who win consistently.

I became fascinated with the game of Blackjack from the moment that I 
understood that its strategy was purely mathematical, based on probability.
It is one of the popular casino games with the smallest house edge,
and is even winnable (mathematically) in some circumstances.  
What is unique about Blackjack is the amount 
of information one can acquire (and exploit) over a number of hands --
there is \emph{memory} to the game.  
Like with many other games, there are many levels of understanding
and calculation that one can achieve.

For the interested player, there are numerous resources available
on the internet and in published texts on the subject of Blackjack 
and its strategies.  
From my informal survey of the literature out there, 
there is enough published information for any player to become
exceptional at Blackjack with practice and discipline.  

So why write a book?  
Writing this book has been an extended exercise in
probability and computing.  
I take personal satisfaction in being able to compute
and verify the results that are already known to many others.
It is just not like me to take the word of experts
on matters that are mathematical.
I need proof.

This book is written for those who appreciate the mathematics of games.  
You'll want to read this if you want to know \emph{how}
strategy is calculated.  
Think of this text as an attempt at compiling a comprehensive
solution manual to the game of Blackjack.  

Why write software?
It is also satisfying to own a set of tools that I can tinker with, 
as opposed to pre-packaged and sourceless programs that are
and difficult to verify.
It is in the spirit of open-source software that I wish to share my work, 
so that others may learn, and even contribute improvements
back to my project.

Finally, I offer some typical words of caution to the confident players.
Even if I were armed with the most advance and sophisticated strategies
that software and math can compute, I would still be averse to gambling.

\begin{itemize}
\item The past not necessarily an indicator of the future.
\item Improvements in expected value are not equivalent to 
	winning with certainty.  
\item There are no guarantees in matters of chance.  
\end{itemize}

I hope you enjoy reading this text as much as I have enjoyed writing it.  

\begin{flushright}
David Fang
\end{flushright}

