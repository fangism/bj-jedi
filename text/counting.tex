% "counting.tex"

\chapter{Card Counting}
\label{sec:counting}

The basic strategy in the previous Chapter~\ref{sec:basic}
assumed (for simplicity) that the probability distribution
of cards being drawn is stationary, when in fact, 
they are actually changing each time a card is revealed and discarded.
The proportion of cards remaining in the deck or shoe
affects the mathematical edges of the game.
It was Edward O.\ Thorp who first quantified the effects of counting in 1962, 
and published his findings on the counting principle,
and subsequently proved it in practice
with his own winnings~\cite{ref:thorp62}.  
Numerous other counting strategies have been devised and analyzed since 
the original findings were 
published~\cite{ref:wong81, ref:uston98, ref:griffin99, ref:carlson10}.

This chapter describes the mathematics behind card counting
and derives some well-known results behind practical counting schemes, 
and quantifies the fluctuations in edges.  
From this point on, we no longer compute using infinite-deck
approximations using the standard deck distribution.  

%%%%%%%%%%%%%%%%%%%%%%%%%%%%%%%%%%%%%%%%%%%%%%%%%%%%%%%%%%%%%%%%%%%%%%%%%%%%%%%
\section{The Dealer's Up-card}
\label{sec:counting:dealer-up}

In Section~{sec:rules:dealer-play:final-pdf}, 
we computed the probability distribution of the 
dealer's final state, given the dealer's up-card (initial state), 
before any player cards are dealt,
assuming an \index{infinite deck}infinite deck.
As an approximation, we neglected the effect of having removed
the dealer's up-card from the deck, and used the same 
standard deck probability distribution (fresh deck) for all sets of 
state transitions.

% % % % % % % % % % % % % % % % % % % % % % % % % % % % % % % % % % % % % % % %
\subsection{Static and Modified Distributions}
\label{sec:counting:dealer-up:static}

One step to improving accuracy is to account for
the change in probabilities after removing the up-card.
However, as an approximation, we still retain the same 
card probability distribution after subsequent cards are removed. 

Let $Mat{D_{R}}$ be the dealer's state transition matrix produced
using a static probability distribution that results from
removing the set of $R$ cards from the deck.  
Then the resulting dealer final state matrix is:

\begin{eqnarray}
\Mat{D_R^*} &=& \LimInf{n}{\Mat{D_R^\prime}\Mat{D_R}^n}
\end{eqnarray}

\TODO:
Show the update table for dealer's expected final state
given reveal card, by removing the up-card from the distribution.
(Still using a static distribution.)  

\TODO:
Show for single deck, 2, 4, compare with infinite deck.

% % % % % % % % % % % % % % % % % % % % % % % % % % % % % % % % % % % % % % % %
\subsection{Exact Odds with Changing Distributions}
\label{sec:counting:dealer-up:exact}

Exact calculation of odds in any process requires
one to account for the effect that \emph{each} removed card has
on the probability distribution.  
This means that evaluating $\Mat{D^*}$ precisely is no longer
a chain-multiplication of constant matrices; 
it requires probability distributions of $\Mat{D}$ to change
at each step.
The computation graph becomes a tree, which can be evaluated
recursively, depth-first, bottom-up.

Recursion stops (tree is pruned) when
dealer reaches a terminal state.
Can also stop when aggregate probability of subtree
falls beneath a threshold.
Beyond threshold, can fall back to an approximate calculation.

\TODO:
Repeat above analysis using exact calculation for 1, 2, 4 decks.  

\TODO:
discuss accuracy and sensitivity of approximation

\TODO:
discuss compute time, tradeoffs.  What is good enough?

%%%%%%%%%%%%%%%%%%%%%%%%%%%%%%%%%%%%%%%%%%%%%%%%%%%%%%%%%%%%%%%%%%%%%%%%%%%%%%%
\section{Sensitivity Analysis}
\label{sec:counting:sensitivity}

Assuming static, basic strategy.

%%%%%%%%%%%%%%%%%%%%%%%%%%%%%%%%%%%%%%%%%%%%%%%%%%%%%%%%%%%%%%%%%%%%%%%%%%%%%%%
\section{Dynamic Strategy}
\label{sec:counting:dynamic}

If the distribution of cards remaining changes, 
shouldn't the strategy also change?
Of course!

