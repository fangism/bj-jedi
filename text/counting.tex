% "counting.tex"

\chapter{Card Counting}
\label{sec:counting}

The basic strategy in the previous Chapter~\ref{sec:basic}
assumed (for simplicity) that the probability distribution
of cards being drawn is stationary, when in fact, 
they are actually changing each time a card is revealed and discarded.
The proportion of cards remaining in the deck or shoe
affects the mathematical edges of the game.
It was Edward Thorp who first quantified the effects of counting in 1962, 
and published his findings on the counting principle,
and subsequently proved it in practice
with his own winnings~\cite{ref:thorp62}.  

This chapter describes the mathematics behind card counting
and derives some well-known results behind practical counting schemes, 
and quantifies the fluctuations in edges.  

\section{Sensitivity Analysis}
\label{sec:counting:sensitivity}

Assuming static, basic strategy.

\section{Dynamic Strategy}
\label{sec:counting:dynamic}

If the distribution of cards remaining changes, 
shouldn't the strategy also change?
Of course!

