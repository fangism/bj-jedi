% "intro.tex"

%%%%%%%%%%%%%%%%%%%%%%%%%%%%%%%%%%%%%%%%%%%%%%%%%%%%%%%%%%%%%%%%%%%%%%%%%%%%%%%
\chapter{Dealer's Process}
\label{sec:dealer}

The dealer's up-card card is essential in determining strategy.
It presents information about the likely progression of the dealer's
hand to a terminal state, which affects the player's expected value.
In this chapter, we evaluate the impact of the dealer's up-card
on the probability spread of final states.  

Table~\ref{tab:dealer-hit-HS17}
summarizes how all non-terminal states will eventually reach
one of the terminal states (a.k.a. absorbing state), 
except for dealer blackjack (denoted BJ), 
which is evaluated separately.
This state-transition process is also known as an
\emph{absorbing Markov chain}\index{Markov chain!absorbing},
since every state must eventually reach an absorbing state.

Table~\ref{tab:dealer-hit-HS17}
compactly represents a sparse state-transition matrices.
With the set of dealers states $\Set{d}$ of size $N_d=\abs{\Set{d}}$,
an expanded $N_d \times N_d$ state-transition matrix $\Mat{D}$ 
would contain entries:

\begin{eqnarray}
\Mat{D}_{i,j} &=&
\begin{cases}
1 & i~\text{is terminal and } i=j \\
0 & i~\text{is terminal and } i\neq{}j \\
\Prob{c} & i~\text{is non-terminal and } \exists k : \Trans{i}{c} = j \\
0 & i~\text{otherwise}
\end{cases}
\end{eqnarray}

\noindent
where $\Prob{c}$ is the probability of drawing card $c$,
based solely on the card distribution.

$D^\prime$ represents the dealer's state transitions
when revealing the hole card,
given that the hole card does not give the dealer a natural or 21.

\begin{eqnarray}
\Mat{D^\prime}_{i,j} &=&
\begin{cases}
1 & i~\text{is terminal and } i=j \\
0 & i~\text{is terminal and } i\neq{}j \\
\Prob{c|\Trans{i}{c}\neq{}\text{BJ}} &
  i~\text{is non-terminal and } \exists k : \Trans{i}{c} = j \\
0 & i~\text{otherwise}
\end{cases}
\end{eqnarray}

\noindent
The only difference between $\Mat{D}$ and $\Mat{D^\prime}$
is in the probability expression.  

% TODO: figure
% TODO: redistribution equation

\section{Probability of Dealer's Terminal States}
\label{sec:dealer-play:final-pdf}

Given the dealer's up-card and state-transition matrix $\Mat{D}$, 
one can calculate the probability of the dealer ending up in each state.
At any given state, the probability distribution of the
next state is simply based on the card probabilities.  
Given an initial state probability unit vector $\Vec{v_i}$ with elements:

\begin{eqnarray}
v_{i,j} &=& 
\begin{cases}
1 & i=j \\
0 & \text{otherwise}
\end{cases}
\end{eqnarray}

\noindent
The probability column vector $\Vec{f_i}$ of the dealer's final states 
can be evaluated as a limit:

\begin{eqnarray}
\Xpose{\Vec{f_i}} &=& \LimInf{n}{\Xpose{\Vec{v_i}}\Mat{D^\prime}\Mat{D}^n} \nonumber \\
&=& \Xpose{\Vec{v_i}} \LimInf{n}{\Mat{D^\prime}\Mat{D}^n} \\
\Mat{D^*} &=& \LimInf{n}{\Mat{D^\prime}\Mat{D}^n} \\
\Vec{f_i} &=& \Paren{\LimInf{n}{\Mat{D^\prime}\Mat{D}^n}} \Vec{v_i} \nonumber \\
&=& \Mat{D^*} \Vec{v_i}
\end{eqnarray}

% TODO: initial identity vector

$\Mat{D^*}$ is the overall dealer final state matrix.  
Since $\Mat{D^*}$ is constant, it can be computed once
for all initial state vectors.
Fortunately, evaluating $\Mat{D^*}$ does not require infinite 
matrix multiplication.
Not only is the dealer's Markov chain absorbing, 
the dealer's hit transition graph
(Table~\ref{tab:dealer-hit-HS17})
is \emph{acyclic}, i.e., no state can ever reach itself again.  
Thus, there exists a topological sort\index{topological sort}
ordering of states that can be used to efficiently and iteratively 
compute the final state probabilities.
The height of the topologically sorted graph is the maximum 
number of iterations needed to converge to terminal states.
For each initial card, the initial probability vector
is a unit vector with all entries set to $0.0$, except for the state
that the initial card represents, which is set to $1.0$.
At each iteration, the probabilities of each state are 
`redistributed'\index{redistribution} to the next states using the 
card probabilities as edge weights, whose sum is $1.0$.  
At any given iteration, the sum of probabilities is always $1.0$.
Probabilities of the terminal states never get redistributed, 
they just accumulate.  
The system converges once all non-terminal state probabilities 
in the probability vector are 0.0.

The process described above assumes an invariant
card distribution, when in fact, as each card is revealed and removed, 
the probability distribution of remaining cards changes.
\index{infinite deck}
Assuming a constant distribution is akin to modeling
the deck as infinite.
For now we proceed using the infinite deck approximation.  
(\textbf{TODO}: Perhaps in a later section, we compute the inaccuracy of
assuming an infinite deck vs. a finite deck, where one 
draws cards without replacement, before player cards are dealt.)

% Pre-peek vs. Post-peek.

\begin{comment}
\begin{table}[ht!]
\caption{Dealer's final state distribution, H17, pre-peek}
\begin{center}
\input tables/dealer-final-H17-pre-peek
\end{center}
\label{tab:dealer-final-H17-pre-peek}
\end{table}

\begin{table}[ht!]
\caption{Dealer's final state distribution, H17, post-peek}
\begin{center}
\input tables/dealer-final-H17-post-peek
\end{center}
\label{tab:dealer-final-H17-post-peek}
\end{table}
\end{comment}

\begin{table}[ht!]
\caption{Dealer's final state distribution, H17}
\begin{center}
\input tables/dealer-final-H17-basic
\end{center}
\label{tab:dealer-final-H17}
\end{table}

\begin{table}[ht!]
\caption{Dealer's final state distribution, S17}
\begin{center}
\input tables/dealer-final-S17-basic
\end{center}
\label{tab:dealer-final-S17}
\end{table}

Suppose that the dealer was given just one up-card with no hole card, 
and the dealer hit until reaching a terminal state ($>=$17 or bust).
Suppose further that if the first two cards are Ace and 10,
then the dealer's hand is treated as a natural blackjack (BJ).
The corresponding spread of the dealer's final states would compute as 
Table~\ref{tab:dealer-final-H17}.
Each row represents the dealer's revealed \emph{up-card}.
The entries are reported in percentages, so each row adds up to 100\%.
The cases where the dealer shows a 10 or Ace are divided into 
\emph{pre-} and \emph{post-peek} entries, as denoted by subscripts.  
In most Blackjack variations, if the up-card is an Ace or 10, 
the dealer will check for blackjack by peeking\index{peek} at the hole card.
In the event of a dealer natural, 
the dealer reveals the hole card (to the player's disappointment)
and the hand ends before the player is given the opportunity to act.

As soon as the player is prompted for action, she may conclude that the 
dealer cannot have a natural.
This means for a revealed Ace, the hole card cannot be 10, 
and for a revealed 10, the hole card cannot be Ace.
Since the first iteration of the redistribution\index{redistribution} 
calculation represents the hole card, the first iteration
should be computed by zero-ing out the probability of the
card that is eliminated from the possibilities, and re-normalizing
the probability of the other cards (to keep a sum of 1.0).  
For all other up-cards, the dealer cannot possibly have a natural,
and thus peeking is irrelevant.

\begin{comment}
\begin{table}[ht!]
\caption{Dealer's final state distribution, S17, pre-peek}
\begin{center}
\input tables/dealer-final-S17-pre-peek
\end{center}
\label{tab:dealer-final-S17-pre-peek}
\end{table}

\begin{table}[ht!]
\caption{Dealer's final state distribution, S17, post-peek}
\begin{center}
\input tables/dealer-final-S17-post-peek
\end{center}
\label{tab:dealer-final-S17-post-peek}
\end{table}
\end{comment}

For the S17 (stand-on-soft-17) variation, the corresponding results are
computed in Table~\ref{tab:dealer-final-S17}.
Observe that compared with Table~\ref{tab:dealer-final-H17},
the entries for rows $\Brace{7,8,9,10}$ are identical because the
soft-17 state (A,6) cannot be reached from $\Brace{7,8,9,10}$.

% observations
From these tables, we can already get some intuition on
which of the dealer's reveal cards are more favorable for the player.
If we look at only the probability of busting and rank them, 
we get from highest to lowest: 6,5,4,3,2,7,8,9,10,A.
There is a significant jump in dealer bust probability from 2 to 7.  
A revealed 6 or 5 is very favorable to the player (over 40\% chance of 
the dealer busting), 
while a revealed A or 10 is very unfavorable 
(only near 20\% chance of dealer busting).
Most strategy guides organize tables with 2-6 on the left, 
and 7-A on the right.

% is peeking good or bad for the player?

When we compute the optimal basic strategy in Chapter~\ref{sec:basic}, 
we will use the post-peek tables, since only they are relevant
when the player has choice in action.  
Pre-peek tables are still useful for evaluating overall player or house
edges \emph{before} each hand is dealt.

